% Options for packages loaded elsewhere
% Options for packages loaded elsewhere
\PassOptionsToPackage{unicode}{hyperref}
\PassOptionsToPackage{hyphens}{url}
\PassOptionsToPackage{dvipsnames,svgnames,x11names}{xcolor}
%
\documentclass[
  norsk,
  a4paper,
]{article}
\usepackage{xcolor}
\usepackage{amsmath,amssymb}
\setcounter{secnumdepth}{5}
\usepackage{iftex}
\ifPDFTeX
  \usepackage[T1]{fontenc}
  \usepackage[utf8]{inputenc}
  \usepackage{textcomp} % provide euro and other symbols
\else % if luatex or xetex
  \usepackage{unicode-math} % this also loads fontspec
  \defaultfontfeatures{Scale=MatchLowercase}
  \defaultfontfeatures[\rmfamily]{Ligatures=TeX,Scale=1}
\fi
\usepackage{lmodern}
\ifPDFTeX\else
  % xetex/luatex font selection
  \setmainfont[]{Times New Roman}
\fi
% Use upquote if available, for straight quotes in verbatim environments
\IfFileExists{upquote.sty}{\usepackage{upquote}}{}
\IfFileExists{microtype.sty}{% use microtype if available
  \usepackage[]{microtype}
  \UseMicrotypeSet[protrusion]{basicmath} % disable protrusion for tt fonts
}{}
\usepackage{setspace}
\makeatletter
\@ifundefined{KOMAClassName}{% if non-KOMA class
  \IfFileExists{parskip.sty}{%
    \usepackage{parskip}
  }{% else
    \setlength{\parindent}{0pt}
    \setlength{\parskip}{6pt plus 2pt minus 1pt}}
}{% if KOMA class
  \KOMAoptions{parskip=half}}
\makeatother
% Make \paragraph and \subparagraph free-standing
\makeatletter
\ifx\paragraph\undefined\else
  \let\oldparagraph\paragraph
  \renewcommand{\paragraph}{
    \@ifstar
      \xxxParagraphStar
      \xxxParagraphNoStar
  }
  \newcommand{\xxxParagraphStar}[1]{\oldparagraph*{#1}\mbox{}}
  \newcommand{\xxxParagraphNoStar}[1]{\oldparagraph{#1}\mbox{}}
\fi
\ifx\subparagraph\undefined\else
  \let\oldsubparagraph\subparagraph
  \renewcommand{\subparagraph}{
    \@ifstar
      \xxxSubParagraphStar
      \xxxSubParagraphNoStar
  }
  \newcommand{\xxxSubParagraphStar}[1]{\oldsubparagraph*{#1}\mbox{}}
  \newcommand{\xxxSubParagraphNoStar}[1]{\oldsubparagraph{#1}\mbox{}}
\fi
\makeatother


\usepackage{longtable,booktabs,array}
\usepackage{calc} % for calculating minipage widths
% Correct order of tables after \paragraph or \subparagraph
\usepackage{etoolbox}
\makeatletter
\patchcmd\longtable{\par}{\if@noskipsec\mbox{}\fi\par}{}{}
\makeatother
% Allow footnotes in longtable head/foot
\IfFileExists{footnotehyper.sty}{\usepackage{footnotehyper}}{\usepackage{footnote}}
\makesavenoteenv{longtable}
\usepackage{graphicx}
\makeatletter
\newsavebox\pandoc@box
\newcommand*\pandocbounded[1]{% scales image to fit in text height/width
  \sbox\pandoc@box{#1}%
  \Gscale@div\@tempa{\textheight}{\dimexpr\ht\pandoc@box+\dp\pandoc@box\relax}%
  \Gscale@div\@tempb{\linewidth}{\wd\pandoc@box}%
  \ifdim\@tempb\p@<\@tempa\p@\let\@tempa\@tempb\fi% select the smaller of both
  \ifdim\@tempa\p@<\p@\scalebox{\@tempa}{\usebox\pandoc@box}%
  \else\usebox{\pandoc@box}%
  \fi%
}
% Set default figure placement to htbp
\def\fps@figure{htbp}
\makeatother


% definitions for citeproc citations
\NewDocumentCommand\citeproctext{}{}
\NewDocumentCommand\citeproc{mm}{%
  \begingroup\def\citeproctext{#2}\cite{#1}\endgroup}
\makeatletter
 % allow citations to break across lines
 \let\@cite@ofmt\@firstofone
 % avoid brackets around text for \cite:
 \def\@biblabel#1{}
 \def\@cite#1#2{{#1\if@tempswa , #2\fi}}
\makeatother
\newlength{\cslhangindent}
\setlength{\cslhangindent}{1.5em}
\newlength{\csllabelwidth}
\setlength{\csllabelwidth}{3em}
\newenvironment{CSLReferences}[2] % #1 hanging-indent, #2 entry-spacing
 {\begin{list}{}{%
  \setlength{\itemindent}{0pt}
  \setlength{\leftmargin}{0pt}
  \setlength{\parsep}{0pt}
  % turn on hanging indent if param 1 is 1
  \ifodd #1
   \setlength{\leftmargin}{\cslhangindent}
   \setlength{\itemindent}{-1\cslhangindent}
  \fi
  % set entry spacing
  \setlength{\itemsep}{#2\baselineskip}}}
 {\end{list}}
\usepackage{calc}
\newcommand{\CSLBlock}[1]{\hfill\break\parbox[t]{\linewidth}{\strut\ignorespaces#1\strut}}
\newcommand{\CSLLeftMargin}[1]{\parbox[t]{\csllabelwidth}{\strut#1\strut}}
\newcommand{\CSLRightInline}[1]{\parbox[t]{\linewidth - \csllabelwidth}{\strut#1\strut}}
\newcommand{\CSLIndent}[1]{\hspace{\cslhangindent}#1}

\ifLuaTeX
\usepackage[bidi=basic]{babel}
\else
\usepackage[bidi=default]{babel}
\fi
\ifPDFTeX
\else
\babelfont{rm}[]{Times New Roman}
\fi
% get rid of language-specific shorthands (see #6817):
\let\LanguageShortHands\languageshorthands
\def\languageshorthands#1{}


\setlength{\emergencystretch}{3em} % prevent overfull lines

\providecommand{\tightlist}{%
  \setlength{\itemsep}{0pt}\setlength{\parskip}{0pt}}



 


\usepackage{fontspec}
\usepackage{multirow}
\usepackage{multicol}
\usepackage{colortbl}
\usepackage{hhline}
\newlength\Oldarrayrulewidth
\newlength\Oldtabcolsep
\usepackage{longtable}
\usepackage{array}
\usepackage{hyperref}
\usepackage{float}
\usepackage{wrapfig}
\usepackage{tabularray}
\usepackage[normalem]{ulem}
\usepackage{graphicx}
\usepackage{rotating}
\UseTblrLibrary{siunitx}
\NewTableCommand{\tinytableDefineColor}[3]{\definecolor{#1}{#2}{#3}}
\newcommand{\tinytableTabularrayUnderline}[1]{\underline{#1}}
\newcommand{\tinytableTabularrayStrikeout}[1]{\sout{#1}}
\makeatletter
\@ifpackageloaded{caption}{}{\usepackage{caption}}
\AtBeginDocument{%
\ifdefined\contentsname
  \renewcommand*\contentsname{Innholdsfortegnelse}
\else
  \newcommand\contentsname{Innholdsfortegnelse}
\fi
\ifdefined\listfigurename
  \renewcommand*\listfigurename{Figuroversikt}
\else
  \newcommand\listfigurename{Figuroversikt}
\fi
\ifdefined\listtablename
  \renewcommand*\listtablename{Tabelloversikt}
\else
  \newcommand\listtablename{Tabelloversikt}
\fi
\ifdefined\figurename
  \renewcommand*\figurename{Figur}
\else
  \newcommand\figurename{Figur}
\fi
\ifdefined\tablename
  \renewcommand*\tablename{Tabell}
\else
  \newcommand\tablename{Tabell}
\fi
}
\@ifpackageloaded{float}{}{\usepackage{float}}
\floatstyle{ruled}
\@ifundefined{c@chapter}{\newfloat{codelisting}{h}{lop}}{\newfloat{codelisting}{h}{lop}[chapter]}
\floatname{codelisting}{Liste}
\newcommand*\listoflistings{\listof{codelisting}{Listeoversikt}}
\makeatother
\makeatletter
\makeatother
\makeatletter
\@ifpackageloaded{caption}{}{\usepackage{caption}}
\@ifpackageloaded{subcaption}{}{\usepackage{subcaption}}
\makeatother
\usepackage{bookmark}
\IfFileExists{xurl.sty}{\usepackage{xurl}}{} % add URL line breaks if available
\urlstyle{same}
\hypersetup{
  pdftitle={Assignment 4, MSB104-gruppe 6, 2025},
  pdfauthor={Karin Liang},
  pdflang={nb},
  colorlinks=true,
  linkcolor={blue},
  filecolor={Maroon},
  citecolor={Blue},
  urlcolor={Blue},
  pdfcreator={LaTeX via pandoc}}


\title{Assignment 4, MSB104-gruppe 6, 2025}
\author{Karin Liang}
\date{}
\begin{document}
\maketitle


\setstretch{1.5}
\section{Kort sammendrag}\label{kort-sammendrag}

Denne oppgaven undersøker regionale forskjeller i økonomisk utvikling og
inntektsulikhet i Belgia, Nederland, Bulgaria og Norge ved hjelp av
Eurostat-data. Analysene kombinerer tre tilnærminger: deskriptive
analyser av BNP per innbygger og Gini, tverrsnittsregresjoner basert på
2017-data, samt alternative funksjonelle former og panelmodeller for
perioden 2008 til 2022. Formålet er å vurdere hvordan økonomisk
utvikling, utdanningsnivå, transportinfrastruktur og demografisk
struktur henger sammen med regionale ulikhetsnivåer.

Resultatene viser at utdanningsnivå og veitetthet er de mest konsistente
forklaringsfaktorene for variasjon i inntektsulikhet, mens økonomisk
vekst har svak og ustabil betydning på tvers av modellene.
Panelestimeringene viser svært begrenset tidsvariasjon i utdanningsnivå
og demografisk struktur, noe som gjør det vanskelig å identifisere
tydelige tidsdynamiske effekter i regional ulikhet. Analysene gir et
oversiktsbilde av hvordan et lite sett av variabler henger sammen med
regional inntektsulikhet, og kan fungere som et utgangspunkt for videre
empiriske analyser og vurderinger innen regionalpolitikken.

\section{Innledning}\label{innledning}

\subsection{Bakgrunn}\label{bakgrunn}

Det finnes fortsatt tydelige forskjeller i økonomisk kapasitet mellom
europeiske land og mellom regioner innen hvert land. Basert på BNP per
innbygger data hentet fra Eurostat kan man se at både på nasjonalt og
regionalt nivå viser økonomisk utvikling betydelig variasjon, både i
langsiktige trender og i tverrsnittssammenligninger. Disse forskjellene
gjenspeiler ikke bare variasjoner i regional produksjonskapasitet og
økonomisk struktur, men er også nært knyttet til flere nøkkelfaktorer,
blant annet utdanningsnivå (høy utdanning), transportinfrastruktur
(veitetthet), demografisk struktur (andel av befolkningen som er 65 år
eller eldre), og nivået på eller endringen i BNP per innbygger.

\subsection{Mål}\label{muxe5l}

Denne oppgaven integrerer analysene fra oppgave 1, 2 og 3 for
systematisk å undersøke regionale forskjeller i økonomisk utvikling og
forholdet mellom regional utvikling og inntektsulikhet i Belgia (BE),
Nederland (NL), Bulgaria (BG) og Norge (NO). Oppgavene fokuserer på
sentrale faktorer som økonomisk utvikling, utdanningsnivå,
transportinfrastruktur og demografisk struktur, og analyserer hvordan
disse faktorene henger sammen med regional inntektsulikhet. Videre
vurderes hvor robuste disse sammenhengene er på tvers av ulike
datastrukturer og metodiske rammeverk.

Hovedmålet i oppgave 1 er å benytte BNP og befolkningsdata fra Eurostat
til å konstruere BNP per innbygger (BNPC) og Gini koeffisienten på NUTS2
nivå for perioden 2000 til 2023 for de fire landene. Dette brukes til å
sammenligne nivået på regional økonomisk utvikling og de langsiktige
utviklingstrendene. Gjennom en beskrivende analyse presenteres
utviklingsmønstre for regionene, noe som gir et viktig bakteppe for
videre analyser av regionale forskjeller og inntektsulikhet.

Oppgave 2 undersøker tverrsnittssammenhengen mellom regional utvikling
og inntektsulikhet. Basert på tverrsnittsdata for 2017 estimeres en
multiple lineær regresjonsmodell for å vurdere om økonomisk utvikling
(endring i BNP per innbygger), transportinfrastruktur (veitetthet),
utdanningsnivå (høy utdanning) og demografisk struktur (andel 65 år
eller eldre) er systematisk forbundet med regionale nivåer av Gini
koeffisienten.

Oppgave 3 utvider analysen metodisk ved å benytte alternative
funksjonelle former, som log lineær modell, kvadratisk modell og kubisk
modell, samt panelmodeller med faste effekter. Hensikten er å undersøke
om de samme variablene viser robuste sammenhenger under ulike
modellspecifikasjoner og i tidsdimensjonen. Paneldatastrukturen gjør det
mulig å kontrollere for uobserverbare, tidsinvariante regionale
egenskaper, noe som gir et mer helhetlig grunnlag for å analysere de
dynamiske forholdene mellom regional utvikling og inntektsulikhet.

\subsection{Betydning}\label{betydning}

Forholdet mellom regional utvikling og inntektsulikhet er viktig å
studere fordi økonomiske forskjeller mellom regioner påvirker
inntektsnivået, levekårene og de langsiktige mulighetene for
befolkningen. Å identifisere hvilke faktorer som henger sammen med
ulikhet, som BNP per innbygger, utdanningsnivå, veitetthet eller
demografisk struktur, gir grunnlag for mer treffsikre regionale
politiske tiltak.

Ved å kombinere langtidsdata for BNP per innbygger fra oppgave 1,
tverrsnittsresultater fra oppgave 2 og alternative funksjonelle former
og faste effektsmodeller fra oppgave 3, undersøker disse oppgavene om de
viktigste faktorene bak regional inntektsulikhet er stabile og
pålitelige. Denne helhetlige tilnærmingen gir en bredere forståelse av
forholdet mellom regional utvikling og inntektsulikhet, og kan bidra med
et empirisk grunnlag for videre politiske vurderinger.

\section{Litteraturgjennomgang}\label{litteraturgjennomgang}

\subsection{Tidligere arbeid}\label{tidligere-arbeid}

Denne delen gir en oversikt over relevant litteratur fra oppgave 1--3 og
danner et faglig grunnlag for å forstå sammenhenger mellom regional
utvikling, utdanning, infrastruktur, demografi og inntektsulikhet.

Lessmann \& Seidel (2017) utnytter satellittdata for nattlys fra 180
land og 3166 regioner, og finner at om lag 67--70 \% av landene viser
sigma-konvergens, der inntektsforskjellene mellom regioner har blitt
mindre over tid. Artikkelen avdekker også en N-formet sammenheng mellom
økonomisk utvikling og regional ulikhet: ulikheten kan først øke,
deretter avta, og igjen stige i høyinntektsfaser. Videre peker
artikkelen på at utvikling, mobilitet, åpenhet, ressurser,
institusjoner, overføringer og utdanning, og etnisitet utgjør syv
grupper av faktorer som kan påvirke determinanter for regional ulikhet.
Denne artikkelen gir et teoretisk grunnlag for disse oppgavene og viser
at regional ulikhet påvirkes av flere strukturelle forhold samtidig.

I oppgave 1 benyttes Cappelen \& Mjøset (2009) til å gi ytterligere
forklaringer på hvordan regionale økonomiske forskjeller oppstår.
Cappelen \& Mjøset (2009) bruker ressursforvaltning og økonomisk
utvikling i Norge, og viser at naturressurser, næringsstruktur og
institusjonell utforming påvirker regioners økonomiske
utviklingsmuligheter. Dette innebærer at ressursrike regioner kan ha en
annen vekstbane enn andre regioner, noe som igjen påvirker regional
utvikling. Videre gir Bandeira Morais et al. (2021) en presis
introduksjon til Gini-koeffisienten, og understreker at Gini er et av de
mest brukte målene for økonomisk ulikhet, der verdien går fra 0 (full
likhet) til 1 (full ulikhet). Dette gir et teoretisk grunnlag for å
bruke Gini-koeffisienten som hovedindikator for regional inntektsulikhet
i disse oppgavene.

I oppgave 2 brukes utdanning, infrastruktur og demografi som sentrale
forklaringsvariabler for regional ulikhet. Coady \& Dizioli (2017) viser
at utdanningsulikhet har en positiv sammenheng med inntektsulikhet, og
at økte utdanningsmuligheter kan redusere forskjeller. Dermed vil
regioner med høyere utdanningsnivå ofte ha lavere ulikhet. Når det
gjelder infrastruktur, finner Calderón \& Servén (2004) at forbedringer
i blant annet veitetthet bidrar til økonomisk vekst og reduserer
regional ulikhet, og at veitetthet typisk har en negativ sammenheng med
Gini. Innen demografi viser Dolls et al. (2019) at aldring kan øke
ulikhet, ettersom en høyere andel eldre kan bidra til større
inntektsforskjeller.

I oppgave 3 viser Kuznets (2019) utviklingen i forholdet mellom
økonomisk vekst og inntektsfordeling, og argumenterer for at ulikhet kan
følge et mønster der den først øker, deretter stabiliserer seg og til
slutt faller, med vekt på at ulikhetsnivået kan endres med økonomiske
strukturer og utviklingsstadier.

Samlet sett bidrar disse litteraturene, gjennom perspektiver på
regionale økonomiske forskjeller, målemetoder for ulikhet og faktorer
som utdanning, infrastruktur og demografi, til det teoretiske
utgangspunktet for disse oppgavene. Litteraturen peker gjennomgående på
at ulikhet påvirkes av flere strukturelle forhold, og at disse kan
variere betydelig mellom regioner. De nevnte litteraturene danner et
faglig grunnlag for å analysere forholdet mellom regional utvikling og
inntektsulikhet.

\subsection{Forskningshull}\label{forskningshull}

Selv om eksisterende litteratur gir et viktig teoretisk grunnlag for å
forstå forholdet mellom regional utvikling og inntektsulikhet, finnes
det fortsatt områder som er lite utforsket.

For det første, på datanivå, bygger tidligere arbeid som Lessmann \&
Seidel (2017) hovedsakelig på nasjonale eller globale sammenligninger og
analyserer sammenhengen mellom utvikling og ulikhet ved hjelp av
tverrnasjonale datasett. Artikkelen benytter ikke Eurostat sine
harmoniserte og direkte sammenlignbare NUTS2-regiondata innenfor samme
statistiske rammeverk.

For det andre, fokuserer mange artikler ofte på en enkelt faktor, som
utdanning, infrastruktur eller demografi, og analyserer sjelden disse
variablene samtidig innenfor samme empiriske ramme. I disse oppgavene
analyseres imidlertid utdanningsnivå (høy utdanning),
transportinfrastruktur (veitetthet), demografisk struktur (65 år eller
eldre) og økonomisk utvikling (BNP per innbygger eller endringen i BNP
per innbygger) samlet i en modell i forhold til regional inntektsulikhet
(Gini).

For det tredje, bruker tidligere arbeid en enkelt analysemodell, enten
tverrsnittsanalyser eller tidsseriestudier, og kombinerer sjelden flere
tilnærminger som langtidsbeskrivelser, tverrsnittsregresjoner og
paneldataanalyse. Dermed blir robuste tester av de samme sammenhengene
på tvers av ulike metoder ofte fraværende.

Disse oppgavene benytter Eurostat-data og gjennomfører en sammenlignende
analyse av Belgia, Bulgaria, Nederland og Norge. Gjennom
langtidsbeskrivelsen av BNP per innbygger i oppgave 1,
tverrsnittsanalysen i oppgave 2, og de alternative funksjonelle formene
samt panelmodellene i oppgave 3, analyseres sammenhengene mellom
regional økonomisk utvikling (BNP per innbygger eller endringen i BNP
per innbygger), utdanning (høy utdanning), transportinfrastruktur
(veitetthet), demografisk struktur (65 år eller eldre) og regional
inntektsulikhet (Gini) fra flere perspektiver. På denne måten gir
oppgavene en helhetlig tilnærming der de samme variablene undersøkes fra
ulike metodiske vinkler, basert på én konsistent datakilde.

\section{Data og metodikk}\label{data-og-metodikk}

\subsection{Datakilder}\label{datakilder}

Datasettet som brukes i oppgave 1 kommer fra Eurostat og inkluderer
``nama\_10r\_3gdp'' (BNP) og ``demo\_r\_pjanaggr3'' (befolkning), begge
tilgjengelig på NUTS3-nivå. Basert på BNP- og befolkningsdata for
perioden 2000--2023 beregnes BNP per innbygger (BNPC), og flere
NUTS3-regioner slås sammen til NUTS2 ved hjelp av befolkningsvekting.
Innenfor samme sammenslåingsramme beregnes også Gini-koeffisienten på
NUTS2-nivå for å måle regional inntektsulikhet. Siden Eurostat kun
dekker Norge fra og med 2010, er tilgjengelige år for Norge kortere enn
for de andre landene i disse oppgavene. I tillegg, selv uten å slette
manglende verdier, varierer antall tilgjengelige observasjoner mellom
land i de opprinnelige Eurostat-datasettene, og sammenslåing etter
region fører også til tap av enkelte observasjoner. Noen NUTS2-regioner
består kun av en NUTS3-enhet, noe som genererer flere NA-verdier ved
beregning av Gini.

Oppgave 2 benytter 2017-data hentet fra datasett brukt i oppgave 1, og
henter i tillegg nye datasett fra Eurostat: utdanning (høy utdanning,
ISCED 5--8) fra ``edat\_lfse\_04'', demografisk struktur (andel av
befolkningen som er 65 år eller eldre) fra ``demo\_r\_pjanaggr3'', for
transportinfrastruktur hentes veilengde fra ``tran\_r\_net'', og
arealdata fra ``reg\_area3'', som kombineres for å beregne veitetthet.
Disse variablene utgjør 2017 tverrsnittsgrunnlaget for analysen av
regional inntektsulikhet. Datasettene varierer i antall observasjoner
mellom land, og enkelte regioner har manglende verdier, noe som fører
til redusert utvalgsstørrelse etter datarensing.

Datasettet som brukes i oppgave 3 dekker perioden 2008--2022 og benytter
de samme variablene som i oppgave 1 og oppgave 2, hentet fra de samme
Eurostat-kildene. Det konstrueres analysefiler på tvers av år og
modeller basert på samme datagrunnlag. Siden tilgjengelige år og
datadekning varierer mellom land, får panelet en ubalansert struktur
(unbalanced panel), og enkelte regioner mangler observasjoner i enkelte
år. Dette reduserer antallet gyldige årsobservasjoner som kan inngå i
estimeringen.

\subsection{Metodisk tilnærming}\label{metodisk-tilnuxe6rming}

Oppgave 1 benytter en beskrivende metode (beskrivende analyse) ved å
beregne og sammenligne BNP per innbygger (BNPC) og Gini-koeffisienten på
NUTS2-nivå for perioden 2000--2023. Dette brukes for å observere og
sammenligne regional utvikling og ulikhet i Belgia, Bulgaria, Nederland
og Norge. Gjennom deskriptiv statistikk og visualisering, der blant
annet figurene fordeling av Gini (NUTS2) etter land og Gini per NUTS2
over tid (fasettert etter land), synliggjøres ulikhetsfordelingen mellom
regioner og utviklingen over tid på en tydelig måte. Den beskrivende
analysen er hensiktsmessig som et første steg før modellering, ettersom
den gir en oversikt over langsiktige mønstre i regional økonomisk
utvikling og identifiserer sentrale forskjeller mellom regioner.

Oppgave 2 bygger på tverrsnittsdata for 2017 og estimerer både en enkel
lineær regresjonsmodell og en multiple lineær regresjonsmodell. I den
enkle lineære regresjonsmodellen testes hovedantakelsene i OLS,
inkludert linearitet, uavhengighet, homoskedastisitet, fravær av
multikollinearitet og normalfordeling av residualer. Deretter inkluderes
nye forklaringsvariabler i den multiple modellen, blant annet høyere
utdanning, veitetthet og andelen av befolkningen som er 65 år eller
eldre. Tverrsnittstilnærmingen gjør det mulig å kontrollere flere
faktorer på samme tidspunkt og sammenligne ulikhetsnivåer mellom
regioner, noe som gir grunnlag for å analysere statistiske sammenhenger
mellom forklaringsvariabler og regional inntektsulikhet.

Oppgave 3 benytter paneldata for perioden 2008--2022 og inkluderer
delmengdeanalyse, alternative funksjonelle former som log-lineær modell,
kvadratisk modell og kubisk modell, heteroskedastisitetstesting, samt
panelestimering (region FE, year FE, two-way FE og country FE). De
alternative funksjonelle formene brukes for å undersøke om forholdet
mellom variabler er robust under ulike modellspecifikasjoner. Deretter
estimeres faste effekter (FE) for å kontrollere for region-, år- eller
lands-spesifikke egenskaper som ikke varierer over tid, slik at modellen
utnytter variasjon på tvers av år for å estimere sammenhengen mellom
økonomisk utvikling, høyere utdanning, veitetthet og andelen av
befolkningen som er 65 år eller eldre og regional inntektsulikhet
(Gini). Fordelen med panel-FE-modeller er at de reduserer skjevhet fra
uobserverbare faktorer og dermed gir mer robuste estimater.

\section{Empiriske funn}\label{empiriske-funn}

\subsection{Resultater fra oppgave 2}\label{resultater-fra-oppgave-2}

\subsubsection{Enkel lineær
regresjonsmodell}\label{enkel-lineuxe6r-regresjonsmodell}

\global\setlength{\Oldarrayrulewidth}{\arrayrulewidth}

\global\setlength{\Oldtabcolsep}{\tabcolsep}

\setlength{\tabcolsep}{2pt}

\renewcommand*{\arraystretch}{0.8}



\providecommand{\ascline}[3]{\noalign{\global\arrayrulewidth #1}\arrayrulecolor[HTML]{#2}\cline{#3}}

\begin{longtable}[c]{|p{0.98in}|p{0.59in}|p{1.08in}|p{0.59in}|p{0.59in}|p{0.40in}}

\caption{\label{tbl-simp-reg}Resultater fra enkel regresjon.}

\tabularnewline

\ascline{1.5pt}{666666}{1-6}

\multicolumn{1}{>{\raggedright}m{\dimexpr 0.98in+0\tabcolsep}}{\textcolor[HTML]{000000}{\fontsize{9}{6}\selectfont{\global\setmainfont{Helvetica}{}}}} & \multicolumn{1}{>{\raggedleft}m{\dimexpr 0.59in+0\tabcolsep}}{\textcolor[HTML]{000000}{\fontsize{9}{6}\selectfont{\global\setmainfont{Helvetica}{Estimate}}}} & \multicolumn{1}{>{\raggedleft}m{\dimexpr 1.08in+0\tabcolsep}}{\textcolor[HTML]{000000}{\fontsize{9}{6}\selectfont{\global\setmainfont{Helvetica}{Standard\ Error}}}} & \multicolumn{1}{>{\raggedleft}m{\dimexpr 0.59in+0\tabcolsep}}{\textcolor[HTML]{000000}{\fontsize{9}{6}\selectfont{\global\setmainfont{Helvetica}{t\ value}}}} & \multicolumn{1}{>{\raggedleft}m{\dimexpr 0.59in+0\tabcolsep}}{\textcolor[HTML]{000000}{\fontsize{9}{6}\selectfont{\global\setmainfont{Helvetica}{Pr(>|t|)}}}} & \multicolumn{1}{>{\raggedright}m{\dimexpr 0.4in+0\tabcolsep}}{\textcolor[HTML]{000000}{\fontsize{9}{6}\selectfont{\global\setmainfont{Helvetica}{}}}} \\

\ascline{1.5pt}{666666}{1-6}\endfirsthead 

\ascline{1.5pt}{666666}{1-6}

\multicolumn{1}{>{\raggedright}m{\dimexpr 0.98in+0\tabcolsep}}{\textcolor[HTML]{000000}{\fontsize{9}{6}\selectfont{\global\setmainfont{Helvetica}{}}}} & \multicolumn{1}{>{\raggedleft}m{\dimexpr 0.59in+0\tabcolsep}}{\textcolor[HTML]{000000}{\fontsize{9}{6}\selectfont{\global\setmainfont{Helvetica}{Estimate}}}} & \multicolumn{1}{>{\raggedleft}m{\dimexpr 1.08in+0\tabcolsep}}{\textcolor[HTML]{000000}{\fontsize{9}{6}\selectfont{\global\setmainfont{Helvetica}{Standard\ Error}}}} & \multicolumn{1}{>{\raggedleft}m{\dimexpr 0.59in+0\tabcolsep}}{\textcolor[HTML]{000000}{\fontsize{9}{6}\selectfont{\global\setmainfont{Helvetica}{t\ value}}}} & \multicolumn{1}{>{\raggedleft}m{\dimexpr 0.59in+0\tabcolsep}}{\textcolor[HTML]{000000}{\fontsize{9}{6}\selectfont{\global\setmainfont{Helvetica}{Pr(>|t|)}}}} & \multicolumn{1}{>{\raggedright}m{\dimexpr 0.4in+0\tabcolsep}}{\textcolor[HTML]{000000}{\fontsize{9}{6}\selectfont{\global\setmainfont{Helvetica}{}}}} \\

\ascline{1.5pt}{666666}{1-6}\endhead



\multicolumn{1}{>{\raggedright}m{\dimexpr 0.98in+0\tabcolsep}}{\textcolor[HTML]{000000}{\fontsize{9}{3}\selectfont{\global\setmainfont{Helvetica}{(Intercept)}}}} & \multicolumn{1}{>{\raggedleft}m{\dimexpr 0.59in+0\tabcolsep}}{\textcolor[HTML]{000000}{\fontsize{9}{3}\selectfont{\global\setmainfont{Helvetica}{0.033}}}} & \multicolumn{1}{>{\raggedleft}m{\dimexpr 1.08in+0\tabcolsep}}{\textcolor[HTML]{000000}{\fontsize{9}{3}\selectfont{\global\setmainfont{Helvetica}{0.019}}}} & \multicolumn{1}{>{\raggedleft}m{\dimexpr 0.59in+0\tabcolsep}}{\textcolor[HTML]{000000}{\fontsize{9}{3}\selectfont{\global\setmainfont{Helvetica}{1.748}}}} & \multicolumn{1}{>{\raggedleft}m{\dimexpr 0.59in+0\tabcolsep}}{\textcolor[HTML]{000000}{\fontsize{9}{3}\selectfont{\global\setmainfont{Helvetica}{0.0951}}}} & \multicolumn{1}{>{\raggedright}m{\dimexpr 0.4in+0\tabcolsep}}{\textcolor[HTML]{000000}{\fontsize{9}{3}\selectfont{\global\setmainfont{Helvetica}{\ \ .}}}} \\





\multicolumn{1}{>{\raggedright}m{\dimexpr 0.98in+0\tabcolsep}}{\textcolor[HTML]{000000}{\fontsize{9}{3}\selectfont{\global\setmainfont{Helvetica}{change\_GDPC\_pct}}}} & \multicolumn{1}{>{\raggedleft}m{\dimexpr 0.59in+0\tabcolsep}}{\textcolor[HTML]{000000}{\fontsize{9}{3}\selectfont{\global\setmainfont{Helvetica}{0.010}}}} & \multicolumn{1}{>{\raggedleft}m{\dimexpr 1.08in+0\tabcolsep}}{\textcolor[HTML]{000000}{\fontsize{9}{3}\selectfont{\global\setmainfont{Helvetica}{0.003}}}} & \multicolumn{1}{>{\raggedleft}m{\dimexpr 0.59in+0\tabcolsep}}{\textcolor[HTML]{000000}{\fontsize{9}{3}\selectfont{\global\setmainfont{Helvetica}{2.906}}}} & \multicolumn{1}{>{\raggedleft}m{\dimexpr 0.59in+0\tabcolsep}}{\textcolor[HTML]{000000}{\fontsize{9}{3}\selectfont{\global\setmainfont{Helvetica}{0.0084}}}} & \multicolumn{1}{>{\raggedright}m{\dimexpr 0.4in+0\tabcolsep}}{\textcolor[HTML]{000000}{\fontsize{9}{3}\selectfont{\global\setmainfont{Helvetica}{\ **}}}} \\

\ascline{1.5pt}{666666}{1-6}



\multicolumn{6}{>{\raggedleft}m{\dimexpr 4.24in+10\tabcolsep}}{\textcolor[HTML]{000000}{\fontsize{9}{3}\selectfont{\global\setmainfont{Helvetica}{\textit{Signif.\ codes:\ 0\ <=\ '***'\ <\ 0.001\ <\ '**'\ <\ 0.01\ <\ '*'\ <\ 0.05}}}}} \\





\multicolumn{6}{>{\raggedright}m{\dimexpr 4.24in+10\tabcolsep}}{\textcolor[HTML]{000000}{\fontsize{9}{3}\selectfont{\global\setmainfont{Helvetica}{}}}} \\





\multicolumn{6}{>{\raggedright}m{\dimexpr 4.24in+10\tabcolsep}}{\textcolor[HTML]{000000}{\fontsize{9}{3}\selectfont{\global\setmainfont{Helvetica}{Residual\ standard\ error:\ 0.04598\ on\ 21\ degrees\ of\ freedom}}}} \\





\multicolumn{6}{>{\raggedright}m{\dimexpr 4.24in+10\tabcolsep}}{\textcolor[HTML]{000000}{\fontsize{9}{3}\selectfont{\global\setmainfont{Helvetica}{Multiple\ R-squared:\ 0.2869,\ Adjusted\ R-squared:\ 0.2529}}}} \\





\multicolumn{6}{>{\raggedright}m{\dimexpr 4.24in+10\tabcolsep}}{\textcolor[HTML]{000000}{\fontsize{9}{3}\selectfont{\global\setmainfont{Helvetica}{F-statistic:\ 8.447\ on\ 21\ and\ 1\ DF,\ p-value:\ 0.0084}}}} \\




\end{longtable}

\arrayrulecolor[HTML]{000000}

\global\setlength{\arrayrulewidth}{\Oldarrayrulewidth}

\global\setlength{\tabcolsep}{\Oldtabcolsep}

\renewcommand*{\arraystretch}{1}

Basert på tverrsnittsanalysen i oppgave 2 estimeres først en enkel
lineær regresjonsmodell for å undersøke effekten av økonomisk utvikling
(endringen i BNP per innbygger) på regional inntektsulikhet (Gini).
Resultatene i Tabell~\ref{tbl-simp-reg} viser en signifikant positiv
sammenheng mellom økonomisk utvikling og ulikhet. Koeffisienten for
økonomisk utvikling er 0.0097 og signifikant på 1 \%-nivå (p = 0.0084),
noe som innebærer at regioner med høyere økonomisk vekst ofte har høyere
Gini-koeffisient. Den justerte R²-verdien er 0.2529, som viser at
økonomisk vekst alene har begrenset forklaringskraft for regionale
ulikheter.

\subsubsection{Multippel lineær
regresjonsmodell}\label{multippel-lineuxe6r-regresjonsmodell}

\global\setlength{\Oldarrayrulewidth}{\arrayrulewidth}

\global\setlength{\Oldtabcolsep}{\tabcolsep}

\setlength{\tabcolsep}{2pt}

\renewcommand*{\arraystretch}{0.8}



\providecommand{\ascline}[3]{\noalign{\global\arrayrulewidth #1}\arrayrulecolor[HTML]{#2}\cline{#3}}

\begin{longtable}[c]{|p{0.98in}|p{0.59in}|p{1.08in}|p{0.59in}|p{0.59in}|p{0.40in}}

\caption{\label{tbl-mult-reg}Resultater fra multippel regresjon.}

\tabularnewline

\ascline{1.5pt}{666666}{1-6}

\multicolumn{1}{>{\raggedright}m{\dimexpr 0.98in+0\tabcolsep}}{\textcolor[HTML]{000000}{\fontsize{9}{6}\selectfont{\global\setmainfont{Helvetica}{}}}} & \multicolumn{1}{>{\raggedleft}m{\dimexpr 0.59in+0\tabcolsep}}{\textcolor[HTML]{000000}{\fontsize{9}{6}\selectfont{\global\setmainfont{Helvetica}{Estimate}}}} & \multicolumn{1}{>{\raggedleft}m{\dimexpr 1.08in+0\tabcolsep}}{\textcolor[HTML]{000000}{\fontsize{9}{6}\selectfont{\global\setmainfont{Helvetica}{Standard\ Error}}}} & \multicolumn{1}{>{\raggedleft}m{\dimexpr 0.59in+0\tabcolsep}}{\textcolor[HTML]{000000}{\fontsize{9}{6}\selectfont{\global\setmainfont{Helvetica}{t\ value}}}} & \multicolumn{1}{>{\raggedleft}m{\dimexpr 0.59in+0\tabcolsep}}{\textcolor[HTML]{000000}{\fontsize{9}{6}\selectfont{\global\setmainfont{Helvetica}{Pr(>|t|)}}}} & \multicolumn{1}{>{\raggedright}m{\dimexpr 0.4in+0\tabcolsep}}{\textcolor[HTML]{000000}{\fontsize{9}{6}\selectfont{\global\setmainfont{Helvetica}{}}}} \\

\ascline{1.5pt}{666666}{1-6}\endfirsthead 

\ascline{1.5pt}{666666}{1-6}

\multicolumn{1}{>{\raggedright}m{\dimexpr 0.98in+0\tabcolsep}}{\textcolor[HTML]{000000}{\fontsize{9}{6}\selectfont{\global\setmainfont{Helvetica}{}}}} & \multicolumn{1}{>{\raggedleft}m{\dimexpr 0.59in+0\tabcolsep}}{\textcolor[HTML]{000000}{\fontsize{9}{6}\selectfont{\global\setmainfont{Helvetica}{Estimate}}}} & \multicolumn{1}{>{\raggedleft}m{\dimexpr 1.08in+0\tabcolsep}}{\textcolor[HTML]{000000}{\fontsize{9}{6}\selectfont{\global\setmainfont{Helvetica}{Standard\ Error}}}} & \multicolumn{1}{>{\raggedleft}m{\dimexpr 0.59in+0\tabcolsep}}{\textcolor[HTML]{000000}{\fontsize{9}{6}\selectfont{\global\setmainfont{Helvetica}{t\ value}}}} & \multicolumn{1}{>{\raggedleft}m{\dimexpr 0.59in+0\tabcolsep}}{\textcolor[HTML]{000000}{\fontsize{9}{6}\selectfont{\global\setmainfont{Helvetica}{Pr(>|t|)}}}} & \multicolumn{1}{>{\raggedright}m{\dimexpr 0.4in+0\tabcolsep}}{\textcolor[HTML]{000000}{\fontsize{9}{6}\selectfont{\global\setmainfont{Helvetica}{}}}} \\

\ascline{1.5pt}{666666}{1-6}\endhead



\multicolumn{1}{>{\raggedright}m{\dimexpr 0.98in+0\tabcolsep}}{\textcolor[HTML]{000000}{\fontsize{9}{3}\selectfont{\global\setmainfont{Helvetica}{(Intercept)}}}} & \multicolumn{1}{>{\raggedleft}m{\dimexpr 0.59in+0\tabcolsep}}{\textcolor[HTML]{000000}{\fontsize{9}{3}\selectfont{\global\setmainfont{Helvetica}{-0.014}}}} & \multicolumn{1}{>{\raggedleft}m{\dimexpr 1.08in+0\tabcolsep}}{\textcolor[HTML]{000000}{\fontsize{9}{3}\selectfont{\global\setmainfont{Helvetica}{0.117}}}} & \multicolumn{1}{>{\raggedleft}m{\dimexpr 0.59in+0\tabcolsep}}{\textcolor[HTML]{000000}{\fontsize{9}{3}\selectfont{\global\setmainfont{Helvetica}{-0.121}}}} & \multicolumn{1}{>{\raggedleft}m{\dimexpr 0.59in+0\tabcolsep}}{\textcolor[HTML]{000000}{\fontsize{9}{3}\selectfont{\global\setmainfont{Helvetica}{0.9052}}}} & \multicolumn{1}{>{\raggedright}m{\dimexpr 0.4in+0\tabcolsep}}{\textcolor[HTML]{000000}{\fontsize{9}{3}\selectfont{\global\setmainfont{Helvetica}{\ \ \ }}}} \\





\multicolumn{1}{>{\raggedright}m{\dimexpr 0.98in+0\tabcolsep}}{\textcolor[HTML]{000000}{\fontsize{9}{3}\selectfont{\global\setmainfont{Helvetica}{change\_GDPC\_pct}}}} & \multicolumn{1}{>{\raggedleft}m{\dimexpr 0.59in+0\tabcolsep}}{\textcolor[HTML]{000000}{\fontsize{9}{3}\selectfont{\global\setmainfont{Helvetica}{0.011}}}} & \multicolumn{1}{>{\raggedleft}m{\dimexpr 1.08in+0\tabcolsep}}{\textcolor[HTML]{000000}{\fontsize{9}{3}\selectfont{\global\setmainfont{Helvetica}{0.006}}}} & \multicolumn{1}{>{\raggedleft}m{\dimexpr 0.59in+0\tabcolsep}}{\textcolor[HTML]{000000}{\fontsize{9}{3}\selectfont{\global\setmainfont{Helvetica}{2.038}}}} & \multicolumn{1}{>{\raggedleft}m{\dimexpr 0.59in+0\tabcolsep}}{\textcolor[HTML]{000000}{\fontsize{9}{3}\selectfont{\global\setmainfont{Helvetica}{0.0574}}}} & \multicolumn{1}{>{\raggedright}m{\dimexpr 0.4in+0\tabcolsep}}{\textcolor[HTML]{000000}{\fontsize{9}{3}\selectfont{\global\setmainfont{Helvetica}{\ \ .}}}} \\





\multicolumn{1}{>{\raggedright}m{\dimexpr 0.98in+0\tabcolsep}}{\textcolor[HTML]{000000}{\fontsize{9}{3}\selectfont{\global\setmainfont{Helvetica}{utdanning\_hoy}}}} & \multicolumn{1}{>{\raggedleft}m{\dimexpr 0.59in+0\tabcolsep}}{\textcolor[HTML]{000000}{\fontsize{9}{3}\selectfont{\global\setmainfont{Helvetica}{0.004}}}} & \multicolumn{1}{>{\raggedleft}m{\dimexpr 1.08in+0\tabcolsep}}{\textcolor[HTML]{000000}{\fontsize{9}{3}\selectfont{\global\setmainfont{Helvetica}{0.001}}}} & \multicolumn{1}{>{\raggedleft}m{\dimexpr 0.59in+0\tabcolsep}}{\textcolor[HTML]{000000}{\fontsize{9}{3}\selectfont{\global\setmainfont{Helvetica}{2.915}}}} & \multicolumn{1}{>{\raggedleft}m{\dimexpr 0.59in+0\tabcolsep}}{\textcolor[HTML]{000000}{\fontsize{9}{3}\selectfont{\global\setmainfont{Helvetica}{0.0097}}}} & \multicolumn{1}{>{\raggedright}m{\dimexpr 0.4in+0\tabcolsep}}{\textcolor[HTML]{000000}{\fontsize{9}{3}\selectfont{\global\setmainfont{Helvetica}{\ **}}}} \\





\multicolumn{1}{>{\raggedright}m{\dimexpr 0.98in+0\tabcolsep}}{\textcolor[HTML]{000000}{\fontsize{9}{3}\selectfont{\global\setmainfont{Helvetica}{vei\_tetthet}}}} & \multicolumn{1}{>{\raggedleft}m{\dimexpr 0.59in+0\tabcolsep}}{\textcolor[HTML]{000000}{\fontsize{9}{3}\selectfont{\global\setmainfont{Helvetica}{-0.001}}}} & \multicolumn{1}{>{\raggedleft}m{\dimexpr 1.08in+0\tabcolsep}}{\textcolor[HTML]{000000}{\fontsize{9}{3}\selectfont{\global\setmainfont{Helvetica}{0.000}}}} & \multicolumn{1}{>{\raggedleft}m{\dimexpr 0.59in+0\tabcolsep}}{\textcolor[HTML]{000000}{\fontsize{9}{3}\selectfont{\global\setmainfont{Helvetica}{-2.232}}}} & \multicolumn{1}{>{\raggedleft}m{\dimexpr 0.59in+0\tabcolsep}}{\textcolor[HTML]{000000}{\fontsize{9}{3}\selectfont{\global\setmainfont{Helvetica}{0.0393}}}} & \multicolumn{1}{>{\raggedright}m{\dimexpr 0.4in+0\tabcolsep}}{\textcolor[HTML]{000000}{\fontsize{9}{3}\selectfont{\global\setmainfont{Helvetica}{\ \ *}}}} \\





\multicolumn{1}{>{\raggedright}m{\dimexpr 0.98in+0\tabcolsep}}{\textcolor[HTML]{000000}{\fontsize{9}{3}\selectfont{\global\setmainfont{Helvetica}{andelen\_over65}}}} & \multicolumn{1}{>{\raggedleft}m{\dimexpr 0.59in+0\tabcolsep}}{\textcolor[HTML]{000000}{\fontsize{9}{3}\selectfont{\global\setmainfont{Helvetica}{-0.003}}}} & \multicolumn{1}{>{\raggedleft}m{\dimexpr 1.08in+0\tabcolsep}}{\textcolor[HTML]{000000}{\fontsize{9}{3}\selectfont{\global\setmainfont{Helvetica}{0.004}}}} & \multicolumn{1}{>{\raggedleft}m{\dimexpr 0.59in+0\tabcolsep}}{\textcolor[HTML]{000000}{\fontsize{9}{3}\selectfont{\global\setmainfont{Helvetica}{-0.663}}}} & \multicolumn{1}{>{\raggedleft}m{\dimexpr 0.59in+0\tabcolsep}}{\textcolor[HTML]{000000}{\fontsize{9}{3}\selectfont{\global\setmainfont{Helvetica}{0.5163}}}} & \multicolumn{1}{>{\raggedright}m{\dimexpr 0.4in+0\tabcolsep}}{\textcolor[HTML]{000000}{\fontsize{9}{3}\selectfont{\global\setmainfont{Helvetica}{\ \ \ }}}} \\

\ascline{1.5pt}{666666}{1-6}



\multicolumn{6}{>{\raggedleft}m{\dimexpr 4.24in+10\tabcolsep}}{\textcolor[HTML]{000000}{\fontsize{9}{3}\selectfont{\global\setmainfont{Helvetica}{\textit{Signif.\ codes:\ 0\ <=\ '***'\ <\ 0.001\ <\ '**'\ <\ 0.01\ <\ '*'\ <\ 0.05}}}}} \\





\multicolumn{6}{>{\raggedright}m{\dimexpr 4.24in+10\tabcolsep}}{\textcolor[HTML]{000000}{\fontsize{9}{3}\selectfont{\global\setmainfont{Helvetica}{}}}} \\





\multicolumn{6}{>{\raggedright}m{\dimexpr 4.24in+10\tabcolsep}}{\textcolor[HTML]{000000}{\fontsize{9}{3}\selectfont{\global\setmainfont{Helvetica}{Residual\ standard\ error:\ 0.03782\ on\ 17\ degrees\ of\ freedom}}}} \\





\multicolumn{6}{>{\raggedright}m{\dimexpr 4.24in+10\tabcolsep}}{\textcolor[HTML]{000000}{\fontsize{9}{3}\selectfont{\global\setmainfont{Helvetica}{Multiple\ R-squared:\ 0.5861,\ Adjusted\ R-squared:\ 0.4888}}}} \\





\multicolumn{6}{>{\raggedright}m{\dimexpr 4.24in+10\tabcolsep}}{\textcolor[HTML]{000000}{\fontsize{9}{3}\selectfont{\global\setmainfont{Helvetica}{F-statistic:\ 6.019\ on\ 17\ and\ 4\ DF,\ p-value:\ 0.0033}}}} \\




\end{longtable}

\arrayrulecolor[HTML]{000000}

\global\setlength{\arrayrulewidth}{\Oldarrayrulewidth}

\global\setlength{\tabcolsep}{\Oldtabcolsep}

\renewcommand*{\arraystretch}{1}

I den multiple lineære regresjonsmodellen inkluderes flere
forklaringsvariabler, blant annet transportinfrastruktur (veitetthet),
utdanning (høy utdanning) og demografisk struktur (andel av befolkningen
som er 65 år eller eldre). Modellen gjengitt i Tabell~\ref{tbl-mult-reg}
viser en tydelig forbedring i forklaringskraft (justert R² = 0.4888),
noe som indikerer at strukturelle forskjeller mellom regioner spiller en
viktig rolle for regional inntektsulikhet. Resultatene viser at høy
utdanning har en signifikant positiv effekt (koeffisient 0.00433, p =
0.00966), som innebærer at regioner med høyere utdanningsnivå også har
større inntektsforskjeller. Veitetthet har en signifikant negativ
sammenheng med ulikhet (koeffisient --0.0009, p = 0.03934), noe som
viser at regioner med bedre transportinfrastruktur som regel har lavere
ulikhet. Variabelen økonomisk utvikling beholder en svak positiv
sammenheng i den multiple modellen (koeffisient 0.01126, p = 0.05738),
men uten sterk statistisk signifikans. Andelen eldre over 65 år viser
ingen signifikant effekt (p = 0.51634).

Samlet sett viser tverrsnittsanalysen at regional inntektsulikhet ikke
kan forklares av økonomisk utvikling alene. Selv om den enkle lineære
modellen indikerer en positiv sammenheng mellom økonomisk vekst og
ulikhet, forklarer modellen kun om lag en fjerdedel av variasjonen
mellom regioner. Den utvidede multiple modellen har derimot betydelig
høyere forklaringskraft og viser at utdanningsnivå og
transportinfrastruktur har de sterkeste og mest konsistente
sammenhengene med ulikhet, noe som er klart sterkere enn økonomisk
vekst. Sammenligningen av de to modellene viser at regional ulikhet er
et resultat av flere strukturelle forhold, der utdanning og
transportinfrastruktur spiller en langt viktigere rolle enn økonomisk
utvikling alene.

\pagebreak[4]

\subsection{Resultater fra oppgave 3}\label{resultater-fra-oppgave-3}

\subsubsection{Alternative funksjonelle
former}\label{alternative-funksjonelle-former}

\global\setlength{\Oldarrayrulewidth}{\arrayrulewidth}

\global\setlength{\Oldtabcolsep}{\tabcolsep}

\setlength{\tabcolsep}{2pt}

\renewcommand*{\arraystretch}{0.8}



\providecommand{\ascline}[3]{\noalign{\global\arrayrulewidth #1}\arrayrulecolor[HTML]{#2}\cline{#3}}

\begin{longtable}[c]{|p{2.36in}|p{0.79in}|p{0.79in}|p{0.79in}}

\caption{\label{tbl-funksjonsformer}logaritmiske, kvadratiske, kubiske}

\tabularnewline

\ascline{1.5pt}{666666}{1-4}

\multicolumn{1}{>{\raggedright}m{\dimexpr 2.36in+0\tabcolsep}}{\textcolor[HTML]{000000}{\fontsize{9}{6}\selectfont{\global\setmainfont{Helvetica}{\ }}}} & \multicolumn{1}{>{\centering}m{\dimexpr 0.79in+0\tabcolsep}}{\textcolor[HTML]{000000}{\fontsize{9}{6}\selectfont{\global\setmainfont{Helvetica}{Log-lineær}}}\textcolor[HTML]{000000}{\fontsize{9}{6}\selectfont{\global\setmainfont{Helvetica}{\linebreak }}}\textcolor[HTML]{000000}{\fontsize{9}{6}\selectfont{\global\setmainfont{Helvetica}{modell}}}} & \multicolumn{1}{>{\centering}m{\dimexpr 0.79in+0\tabcolsep}}{\textcolor[HTML]{000000}{\fontsize{9}{6}\selectfont{\global\setmainfont{Helvetica}{Kvadratisk}}}\textcolor[HTML]{000000}{\fontsize{9}{6}\selectfont{\global\setmainfont{Helvetica}{\linebreak }}}\textcolor[HTML]{000000}{\fontsize{9}{6}\selectfont{\global\setmainfont{Helvetica}{modell}}}} & \multicolumn{1}{>{\centering}m{\dimexpr 0.79in+0\tabcolsep}}{\textcolor[HTML]{000000}{\fontsize{9}{6}\selectfont{\global\setmainfont{Helvetica}{Kubisk}}}\textcolor[HTML]{000000}{\fontsize{9}{6}\selectfont{\global\setmainfont{Helvetica}{\linebreak }}}\textcolor[HTML]{000000}{\fontsize{9}{6}\selectfont{\global\setmainfont{Helvetica}{modell}}}} \\

\ascline{1.5pt}{666666}{1-4}\endfirsthead 

\ascline{1.5pt}{666666}{1-4}

\multicolumn{1}{>{\raggedright}m{\dimexpr 2.36in+0\tabcolsep}}{\textcolor[HTML]{000000}{\fontsize{9}{6}\selectfont{\global\setmainfont{Helvetica}{\ }}}} & \multicolumn{1}{>{\centering}m{\dimexpr 0.79in+0\tabcolsep}}{\textcolor[HTML]{000000}{\fontsize{9}{6}\selectfont{\global\setmainfont{Helvetica}{Log-lineær}}}\textcolor[HTML]{000000}{\fontsize{9}{6}\selectfont{\global\setmainfont{Helvetica}{\linebreak }}}\textcolor[HTML]{000000}{\fontsize{9}{6}\selectfont{\global\setmainfont{Helvetica}{modell}}}} & \multicolumn{1}{>{\centering}m{\dimexpr 0.79in+0\tabcolsep}}{\textcolor[HTML]{000000}{\fontsize{9}{6}\selectfont{\global\setmainfont{Helvetica}{Kvadratisk}}}\textcolor[HTML]{000000}{\fontsize{9}{6}\selectfont{\global\setmainfont{Helvetica}{\linebreak }}}\textcolor[HTML]{000000}{\fontsize{9}{6}\selectfont{\global\setmainfont{Helvetica}{modell}}}} & \multicolumn{1}{>{\centering}m{\dimexpr 0.79in+0\tabcolsep}}{\textcolor[HTML]{000000}{\fontsize{9}{6}\selectfont{\global\setmainfont{Helvetica}{Kubisk}}}\textcolor[HTML]{000000}{\fontsize{9}{6}\selectfont{\global\setmainfont{Helvetica}{\linebreak }}}\textcolor[HTML]{000000}{\fontsize{9}{6}\selectfont{\global\setmainfont{Helvetica}{modell}}}} \\

\ascline{1.5pt}{666666}{1-4}\endhead



\multicolumn{1}{>{\raggedright}m{\dimexpr 2.36in+0\tabcolsep}}{\textcolor[HTML]{000000}{\fontsize{9}{3}\selectfont{\global\setmainfont{Helvetica}{(Intercept)}}}} & \multicolumn{1}{>{\centering}m{\dimexpr 0.79in+0\tabcolsep}}{\textcolor[HTML]{000000}{\fontsize{9}{3}\selectfont{\global\setmainfont{Helvetica}{-0.12151}}}} & \multicolumn{1}{>{\centering}m{\dimexpr 0.79in+0\tabcolsep}}{\textcolor[HTML]{000000}{\fontsize{9}{3}\selectfont{\global\setmainfont{Helvetica}{-0.14363}}}} & \multicolumn{1}{>{\centering}m{\dimexpr 0.79in+0\tabcolsep}}{\textcolor[HTML]{000000}{\fontsize{9}{3}\selectfont{\global\setmainfont{Helvetica}{-0.07947}}}} \\





\multicolumn{1}{>{\raggedright}m{\dimexpr 2.36in+0\tabcolsep}}{\textcolor[HTML]{000000}{\fontsize{9}{3}\selectfont{\global\setmainfont{Helvetica}{}}}} & \multicolumn{1}{>{\centering}m{\dimexpr 0.79in+0\tabcolsep}}{\textcolor[HTML]{000000}{\fontsize{9}{3}\selectfont{\global\setmainfont{Helvetica}{(-0.93697)}}}} & \multicolumn{1}{>{\centering}m{\dimexpr 0.79in+0\tabcolsep}}{\textcolor[HTML]{000000}{\fontsize{9}{3}\selectfont{\global\setmainfont{Helvetica}{(-1.02523)}}}} & \multicolumn{1}{>{\centering}m{\dimexpr 0.79in+0\tabcolsep}}{\textcolor[HTML]{000000}{\fontsize{9}{3}\selectfont{\global\setmainfont{Helvetica}{(-0.46799)}}}} \\





\multicolumn{1}{>{\raggedright}m{\dimexpr 2.36in+0\tabcolsep}}{\textcolor[HTML]{000000}{\fontsize{9}{3}\selectfont{\global\setmainfont{Helvetica}{log(change\_GDPC\_pct\ +\ 1)}}}} & \multicolumn{1}{>{\centering}m{\dimexpr 0.79in+0\tabcolsep}}{\textcolor[HTML]{000000}{\fontsize{9}{3}\selectfont{\global\setmainfont{Helvetica}{0.08266*}}}} & \multicolumn{1}{>{\centering}m{\dimexpr 0.79in+0\tabcolsep}}{\textcolor[HTML]{000000}{\fontsize{9}{3}\selectfont{\global\setmainfont{Helvetica}{}}}} & \multicolumn{1}{>{\centering}m{\dimexpr 0.79in+0\tabcolsep}}{\textcolor[HTML]{000000}{\fontsize{9}{3}\selectfont{\global\setmainfont{Helvetica}{}}}} \\





\multicolumn{1}{>{\raggedright}m{\dimexpr 2.36in+0\tabcolsep}}{\textcolor[HTML]{000000}{\fontsize{9}{3}\selectfont{\global\setmainfont{Helvetica}{}}}} & \multicolumn{1}{>{\centering}m{\dimexpr 0.79in+0\tabcolsep}}{\textcolor[HTML]{000000}{\fontsize{9}{3}\selectfont{\global\setmainfont{Helvetica}{(2.36393)}}}} & \multicolumn{1}{>{\centering}m{\dimexpr 0.79in+0\tabcolsep}}{\textcolor[HTML]{000000}{\fontsize{9}{3}\selectfont{\global\setmainfont{Helvetica}{}}}} & \multicolumn{1}{>{\centering}m{\dimexpr 0.79in+0\tabcolsep}}{\textcolor[HTML]{000000}{\fontsize{9}{3}\selectfont{\global\setmainfont{Helvetica}{}}}} \\





\multicolumn{1}{>{\raggedright}m{\dimexpr 2.36in+0\tabcolsep}}{\textcolor[HTML]{000000}{\fontsize{9}{3}\selectfont{\global\setmainfont{Helvetica}{vei\_tetthet}}}} & \multicolumn{1}{>{\centering}m{\dimexpr 0.79in+0\tabcolsep}}{\textcolor[HTML]{000000}{\fontsize{9}{3}\selectfont{\global\setmainfont{Helvetica}{-0.00087*}}}} & \multicolumn{1}{>{\centering}m{\dimexpr 0.79in+0\tabcolsep}}{\textcolor[HTML]{000000}{\fontsize{9}{3}\selectfont{\global\setmainfont{Helvetica}{-0.00086*}}}} & \multicolumn{1}{>{\centering}m{\dimexpr 0.79in+0\tabcolsep}}{\textcolor[HTML]{000000}{\fontsize{9}{3}\selectfont{\global\setmainfont{Helvetica}{-0.00078+}}}} \\





\multicolumn{1}{>{\raggedright}m{\dimexpr 2.36in+0\tabcolsep}}{\textcolor[HTML]{000000}{\fontsize{9}{3}\selectfont{\global\setmainfont{Helvetica}{}}}} & \multicolumn{1}{>{\centering}m{\dimexpr 0.79in+0\tabcolsep}}{\textcolor[HTML]{000000}{\fontsize{9}{3}\selectfont{\global\setmainfont{Helvetica}{(-2.24811)}}}} & \multicolumn{1}{>{\centering}m{\dimexpr 0.79in+0\tabcolsep}}{\textcolor[HTML]{000000}{\fontsize{9}{3}\selectfont{\global\setmainfont{Helvetica}{(-2.20387)}}}} & \multicolumn{1}{>{\centering}m{\dimexpr 0.79in+0\tabcolsep}}{\textcolor[HTML]{000000}{\fontsize{9}{3}\selectfont{\global\setmainfont{Helvetica}{(-1.87064)}}}} \\





\multicolumn{1}{>{\raggedright}m{\dimexpr 2.36in+0\tabcolsep}}{\textcolor[HTML]{000000}{\fontsize{9}{3}\selectfont{\global\setmainfont{Helvetica}{andelen\_over65}}}} & \multicolumn{1}{>{\centering}m{\dimexpr 0.79in+0\tabcolsep}}{\textcolor[HTML]{000000}{\fontsize{9}{3}\selectfont{\global\setmainfont{Helvetica}{-0.00236}}}} & \multicolumn{1}{>{\centering}m{\dimexpr 0.79in+0\tabcolsep}}{\textcolor[HTML]{000000}{\fontsize{9}{3}\selectfont{\global\setmainfont{Helvetica}{-0.00046}}}} & \multicolumn{1}{>{\centering}m{\dimexpr 0.79in+0\tabcolsep}}{\textcolor[HTML]{000000}{\fontsize{9}{3}\selectfont{\global\setmainfont{Helvetica}{-0.00003}}}} \\





\multicolumn{1}{>{\raggedright}m{\dimexpr 2.36in+0\tabcolsep}}{\textcolor[HTML]{000000}{\fontsize{9}{3}\selectfont{\global\setmainfont{Helvetica}{}}}} & \multicolumn{1}{>{\centering}m{\dimexpr 0.79in+0\tabcolsep}}{\textcolor[HTML]{000000}{\fontsize{9}{3}\selectfont{\global\setmainfont{Helvetica}{(-0.57973)}}}} & \multicolumn{1}{>{\centering}m{\dimexpr 0.79in+0\tabcolsep}}{\textcolor[HTML]{000000}{\fontsize{9}{3}\selectfont{\global\setmainfont{Helvetica}{(-0.10355)}}}} & \multicolumn{1}{>{\centering}m{\dimexpr 0.79in+0\tabcolsep}}{\textcolor[HTML]{000000}{\fontsize{9}{3}\selectfont{\global\setmainfont{Helvetica}{(-0.00574)}}}} \\





\multicolumn{1}{>{\raggedright}m{\dimexpr 2.36in+0\tabcolsep}}{\textcolor[HTML]{000000}{\fontsize{9}{3}\selectfont{\global\setmainfont{Helvetica}{Høy\_utdanning}}}} & \multicolumn{1}{>{\centering}m{\dimexpr 0.79in+0\tabcolsep}}{\textcolor[HTML]{000000}{\fontsize{9}{3}\selectfont{\global\setmainfont{Helvetica}{0.00467**}}}} & \multicolumn{1}{>{\centering}m{\dimexpr 0.79in+0\tabcolsep}}{\textcolor[HTML]{000000}{\fontsize{9}{3}\selectfont{\global\setmainfont{Helvetica}{0.00475**}}}} & \multicolumn{1}{>{\centering}m{\dimexpr 0.79in+0\tabcolsep}}{\textcolor[HTML]{000000}{\fontsize{9}{3}\selectfont{\global\setmainfont{Helvetica}{0.00450**}}}} \\





\multicolumn{1}{>{\raggedright}m{\dimexpr 2.36in+0\tabcolsep}}{\textcolor[HTML]{000000}{\fontsize{9}{3}\selectfont{\global\setmainfont{Helvetica}{}}}} & \multicolumn{1}{>{\centering}m{\dimexpr 0.79in+0\tabcolsep}}{\textcolor[HTML]{000000}{\fontsize{9}{3}\selectfont{\global\setmainfont{Helvetica}{(3.17124)}}}} & \multicolumn{1}{>{\centering}m{\dimexpr 0.79in+0\tabcolsep}}{\textcolor[HTML]{000000}{\fontsize{9}{3}\selectfont{\global\setmainfont{Helvetica}{(3.26921)}}}} & \multicolumn{1}{>{\centering}m{\dimexpr 0.79in+0\tabcolsep}}{\textcolor[HTML]{000000}{\fontsize{9}{3}\selectfont{\global\setmainfont{Helvetica}{(2.95712)}}}} \\





\multicolumn{1}{>{\raggedright}m{\dimexpr 2.36in+0\tabcolsep}}{\textcolor[HTML]{000000}{\fontsize{9}{3}\selectfont{\global\setmainfont{Helvetica}{change\_GDPC\_pct}}}} & \multicolumn{1}{>{\centering}m{\dimexpr 0.79in+0\tabcolsep}}{\textcolor[HTML]{000000}{\fontsize{9}{3}\selectfont{\global\setmainfont{Helvetica}{}}}} & \multicolumn{1}{>{\centering}m{\dimexpr 0.79in+0\tabcolsep}}{\textcolor[HTML]{000000}{\fontsize{9}{3}\selectfont{\global\setmainfont{Helvetica}{0.03697*}}}} & \multicolumn{1}{>{\centering}m{\dimexpr 0.79in+0\tabcolsep}}{\textcolor[HTML]{000000}{\fontsize{9}{3}\selectfont{\global\setmainfont{Helvetica}{-0.00154}}}} \\





\multicolumn{1}{>{\raggedright}m{\dimexpr 2.36in+0\tabcolsep}}{\textcolor[HTML]{000000}{\fontsize{9}{3}\selectfont{\global\setmainfont{Helvetica}{}}}} & \multicolumn{1}{>{\centering}m{\dimexpr 0.79in+0\tabcolsep}}{\textcolor[HTML]{000000}{\fontsize{9}{3}\selectfont{\global\setmainfont{Helvetica}{}}}} & \multicolumn{1}{>{\centering}m{\dimexpr 0.79in+0\tabcolsep}}{\textcolor[HTML]{000000}{\fontsize{9}{3}\selectfont{\global\setmainfont{Helvetica}{(2.12585)}}}} & \multicolumn{1}{>{\centering}m{\dimexpr 0.79in+0\tabcolsep}}{\textcolor[HTML]{000000}{\fontsize{9}{3}\selectfont{\global\setmainfont{Helvetica}{(-0.02643)}}}} \\





\multicolumn{1}{>{\raggedright}m{\dimexpr 2.36in+0\tabcolsep}}{\textcolor[HTML]{000000}{\fontsize{9}{3}\selectfont{\global\setmainfont{Helvetica}{I(change\_GDPC\_pct\\\textasciicircum 2)}}}} & \multicolumn{1}{>{\centering}m{\dimexpr 0.79in+0\tabcolsep}}{\textcolor[HTML]{000000}{\fontsize{9}{3}\selectfont{\global\setmainfont{Helvetica}{}}}} & \multicolumn{1}{>{\centering}m{\dimexpr 0.79in+0\tabcolsep}}{\textcolor[HTML]{000000}{\fontsize{9}{3}\selectfont{\global\setmainfont{Helvetica}{-0.00200}}}} & \multicolumn{1}{>{\centering}m{\dimexpr 0.79in+0\tabcolsep}}{\textcolor[HTML]{000000}{\fontsize{9}{3}\selectfont{\global\setmainfont{Helvetica}{0.00403}}}} \\





\multicolumn{1}{>{\raggedright}m{\dimexpr 2.36in+0\tabcolsep}}{\textcolor[HTML]{000000}{\fontsize{9}{3}\selectfont{\global\setmainfont{Helvetica}{}}}} & \multicolumn{1}{>{\centering}m{\dimexpr 0.79in+0\tabcolsep}}{\textcolor[HTML]{000000}{\fontsize{9}{3}\selectfont{\global\setmainfont{Helvetica}{}}}} & \multicolumn{1}{>{\centering}m{\dimexpr 0.79in+0\tabcolsep}}{\textcolor[HTML]{000000}{\fontsize{9}{3}\selectfont{\global\setmainfont{Helvetica}{(-1.55222)}}}} & \multicolumn{1}{>{\centering}m{\dimexpr 0.79in+0\tabcolsep}}{\textcolor[HTML]{000000}{\fontsize{9}{3}\selectfont{\global\setmainfont{Helvetica}{(0.45855)}}}} \\





\multicolumn{1}{>{\raggedright}m{\dimexpr 2.36in+0\tabcolsep}}{\textcolor[HTML]{000000}{\fontsize{9}{3}\selectfont{\global\setmainfont{Helvetica}{I(change\_GDPC\_pct\\\textasciicircum 3)}}}} & \multicolumn{1}{>{\centering}m{\dimexpr 0.79in+0\tabcolsep}}{\textcolor[HTML]{000000}{\fontsize{9}{3}\selectfont{\global\setmainfont{Helvetica}{}}}} & \multicolumn{1}{>{\centering}m{\dimexpr 0.79in+0\tabcolsep}}{\textcolor[HTML]{000000}{\fontsize{9}{3}\selectfont{\global\setmainfont{Helvetica}{}}}} & \multicolumn{1}{>{\centering}m{\dimexpr 0.79in+0\tabcolsep}}{\textcolor[HTML]{000000}{\fontsize{9}{3}\selectfont{\global\setmainfont{Helvetica}{-0.00028}}}} \\





\multicolumn{1}{>{\raggedright}m{\dimexpr 2.36in+0\tabcolsep}}{\textcolor[HTML]{000000}{\fontsize{9}{3}\selectfont{\global\setmainfont{Helvetica}{}}}} & \multicolumn{1}{>{\centering}m{\dimexpr 0.79in+0\tabcolsep}}{\textcolor[HTML]{000000}{\fontsize{9}{3}\selectfont{\global\setmainfont{Helvetica}{}}}} & \multicolumn{1}{>{\centering}m{\dimexpr 0.79in+0\tabcolsep}}{\textcolor[HTML]{000000}{\fontsize{9}{3}\selectfont{\global\setmainfont{Helvetica}{}}}} & \multicolumn{1}{>{\centering}m{\dimexpr 0.79in+0\tabcolsep}}{\textcolor[HTML]{000000}{\fontsize{9}{3}\selectfont{\global\setmainfont{Helvetica}{(-0.69392)}}}} \\

\ascline{0.75pt}{666666}{1-4}



\multicolumn{1}{>{\raggedright}m{\dimexpr 2.36in+0\tabcolsep}}{\textcolor[HTML]{000000}{\fontsize{9}{3}\selectfont{\global\setmainfont{Helvetica}{Num.Obs.}}}} & \multicolumn{1}{>{\centering}m{\dimexpr 0.79in+0\tabcolsep}}{\textcolor[HTML]{000000}{\fontsize{9}{3}\selectfont{\global\setmainfont{Helvetica}{22}}}} & \multicolumn{1}{>{\centering}m{\dimexpr 0.79in+0\tabcolsep}}{\textcolor[HTML]{000000}{\fontsize{9}{3}\selectfont{\global\setmainfont{Helvetica}{22}}}} & \multicolumn{1}{>{\centering}m{\dimexpr 0.79in+0\tabcolsep}}{\textcolor[HTML]{000000}{\fontsize{9}{3}\selectfont{\global\setmainfont{Helvetica}{22}}}} \\





\multicolumn{1}{>{\raggedright}m{\dimexpr 2.36in+0\tabcolsep}}{\textcolor[HTML]{000000}{\fontsize{9}{3}\selectfont{\global\setmainfont{Helvetica}{R2}}}} & \multicolumn{1}{>{\centering}m{\dimexpr 0.79in+0\tabcolsep}}{\textcolor[HTML]{000000}{\fontsize{9}{3}\selectfont{\global\setmainfont{Helvetica}{0.612}}}} & \multicolumn{1}{>{\centering}m{\dimexpr 0.79in+0\tabcolsep}}{\textcolor[HTML]{000000}{\fontsize{9}{3}\selectfont{\global\setmainfont{Helvetica}{0.640}}}} & \multicolumn{1}{>{\centering}m{\dimexpr 0.79in+0\tabcolsep}}{\textcolor[HTML]{000000}{\fontsize{9}{3}\selectfont{\global\setmainfont{Helvetica}{0.651}}}} \\





\multicolumn{1}{>{\raggedright}m{\dimexpr 2.36in+0\tabcolsep}}{\textcolor[HTML]{000000}{\fontsize{9}{3}\selectfont{\global\setmainfont{Helvetica}{R2\ Adj.}}}} & \multicolumn{1}{>{\centering}m{\dimexpr 0.79in+0\tabcolsep}}{\textcolor[HTML]{000000}{\fontsize{9}{3}\selectfont{\global\setmainfont{Helvetica}{0.521}}}} & \multicolumn{1}{>{\centering}m{\dimexpr 0.79in+0\tabcolsep}}{\textcolor[HTML]{000000}{\fontsize{9}{3}\selectfont{\global\setmainfont{Helvetica}{0.528}}}} & \multicolumn{1}{>{\centering}m{\dimexpr 0.79in+0\tabcolsep}}{\textcolor[HTML]{000000}{\fontsize{9}{3}\selectfont{\global\setmainfont{Helvetica}{0.512}}}} \\





\multicolumn{1}{>{\raggedright}m{\dimexpr 2.36in+0\tabcolsep}}{\textcolor[HTML]{000000}{\fontsize{9}{3}\selectfont{\global\setmainfont{Helvetica}{AIC}}}} & \multicolumn{1}{>{\centering}m{\dimexpr 0.79in+0\tabcolsep}}{\textcolor[HTML]{000000}{\fontsize{9}{3}\selectfont{\global\setmainfont{Helvetica}{-76.8}}}} & \multicolumn{1}{>{\centering}m{\dimexpr 0.79in+0\tabcolsep}}{\textcolor[HTML]{000000}{\fontsize{9}{3}\selectfont{\global\setmainfont{Helvetica}{-76.4}}}} & \multicolumn{1}{>{\centering}m{\dimexpr 0.79in+0\tabcolsep}}{\textcolor[HTML]{000000}{\fontsize{9}{3}\selectfont{\global\setmainfont{Helvetica}{-75.1}}}} \\





\multicolumn{1}{>{\raggedright}m{\dimexpr 2.36in+0\tabcolsep}}{\textcolor[HTML]{000000}{\fontsize{9}{3}\selectfont{\global\setmainfont{Helvetica}{BIC}}}} & \multicolumn{1}{>{\centering}m{\dimexpr 0.79in+0\tabcolsep}}{\textcolor[HTML]{000000}{\fontsize{9}{3}\selectfont{\global\setmainfont{Helvetica}{-70.2}}}} & \multicolumn{1}{>{\centering}m{\dimexpr 0.79in+0\tabcolsep}}{\textcolor[HTML]{000000}{\fontsize{9}{3}\selectfont{\global\setmainfont{Helvetica}{-68.8}}}} & \multicolumn{1}{>{\centering}m{\dimexpr 0.79in+0\tabcolsep}}{\textcolor[HTML]{000000}{\fontsize{9}{3}\selectfont{\global\setmainfont{Helvetica}{-66.4}}}} \\





\multicolumn{1}{>{\raggedright}m{\dimexpr 2.36in+0\tabcolsep}}{\textcolor[HTML]{000000}{\fontsize{9}{3}\selectfont{\global\setmainfont{Helvetica}{Log.Lik.}}}} & \multicolumn{1}{>{\centering}m{\dimexpr 0.79in+0\tabcolsep}}{\textcolor[HTML]{000000}{\fontsize{9}{3}\selectfont{\global\setmainfont{Helvetica}{44.387}}}} & \multicolumn{1}{>{\centering}m{\dimexpr 0.79in+0\tabcolsep}}{\textcolor[HTML]{000000}{\fontsize{9}{3}\selectfont{\global\setmainfont{Helvetica}{45.209}}}} & \multicolumn{1}{>{\centering}m{\dimexpr 0.79in+0\tabcolsep}}{\textcolor[HTML]{000000}{\fontsize{9}{3}\selectfont{\global\setmainfont{Helvetica}{45.557}}}} \\





\multicolumn{1}{>{\raggedright}m{\dimexpr 2.36in+0\tabcolsep}}{\textcolor[HTML]{000000}{\fontsize{9}{3}\selectfont{\global\setmainfont{Helvetica}{F}}}} & \multicolumn{1}{>{\centering}m{\dimexpr 0.79in+0\tabcolsep}}{\textcolor[HTML]{000000}{\fontsize{9}{3}\selectfont{\global\setmainfont{Helvetica}{6.715}}}} & \multicolumn{1}{>{\centering}m{\dimexpr 0.79in+0\tabcolsep}}{\textcolor[HTML]{000000}{\fontsize{9}{3}\selectfont{\global\setmainfont{Helvetica}{5.696}}}} & \multicolumn{1}{>{\centering}m{\dimexpr 0.79in+0\tabcolsep}}{\textcolor[HTML]{000000}{\fontsize{9}{3}\selectfont{\global\setmainfont{Helvetica}{4.673}}}} \\





\multicolumn{1}{>{\raggedright}m{\dimexpr 2.36in+0\tabcolsep}}{\textcolor[HTML]{000000}{\fontsize{9}{3}\selectfont{\global\setmainfont{Helvetica}{RMSE}}}} & \multicolumn{1}{>{\centering}m{\dimexpr 0.79in+0\tabcolsep}}{\textcolor[HTML]{000000}{\fontsize{9}{3}\selectfont{\global\setmainfont{Helvetica}{0.03}}}} & \multicolumn{1}{>{\centering}m{\dimexpr 0.79in+0\tabcolsep}}{\textcolor[HTML]{000000}{\fontsize{9}{3}\selectfont{\global\setmainfont{Helvetica}{0.03}}}} & \multicolumn{1}{>{\centering}m{\dimexpr 0.79in+0\tabcolsep}}{\textcolor[HTML]{000000}{\fontsize{9}{3}\selectfont{\global\setmainfont{Helvetica}{0.03}}}} \\

\ascline{1.5pt}{666666}{1-4}



\multicolumn{4}{>{\raggedright}m{\dimexpr 4.72in+6\tabcolsep}}{\textcolor[HTML]{000000}{\fontsize{9}{3}\selectfont{\global\setmainfont{Helvetica}{+\ p\ <\ 0.1,\ *\ p\ <\ 0.05,\ **\ p\ <\ 0.01,\ ***\ p\ <\ 0.001}}}} \\





\multicolumn{4}{>{\raggedright}m{\dimexpr 4.72in+6\tabcolsep}}{\textcolor[HTML]{000000}{\fontsize{9}{3}\selectfont{\global\setmainfont{Helvetica}{Viser\ t-verdier\ i\ parentes\ under\ estimatene.}}}} \\




\end{longtable}

\arrayrulecolor[HTML]{000000}

\global\setlength{\arrayrulewidth}{\Oldarrayrulewidth}

\global\setlength{\tabcolsep}{\Oldtabcolsep}

\renewcommand*{\arraystretch}{1}

I Tabell~\ref{tbl-funksjonsformer} presenteres tre alternative
funksjonelle former i oppgave 3: log lineær modell, kvadratisk modell og
kubisk modell. Modellene estimeres med de samme kontrollvariablene som i
oppgave 2, og resultatene er oppsummert i tabell
Tabell~\ref{tbl-funksjonsformer}.

For det første viser de tre modellene R2 verdier på henholdsvis 0.612
for log modellen, 0.640 for den kvadratiske modellen og 0.651 for den
kubiske modellen. Den justerte R2 ligger i området 0.512 til 0.528, med
små forskjeller mellom modellene. Når det gjelder AIC, er log modellen
lavest med verdien -76.8, tett fulgt av den kvadratiske modellen med
-76.4, mens den kubiske modellen er noe høyere med -75.1. Samlet sett
fremstår log modellen som den beste etter modellvalgsindikatorene, mens
den kvadratiske modellen har noe høyere forklaringskraft.

For det andre viser den log lineære modellen en koeffisient på 0.08266
for log(change\_GDPC\_pct + 1), som er statistisk signifikant og
indikerer en positiv sammenheng mellom økonomisk utvikling og regional
ulikhet. I den kvadratiske modellen er koeffisienten for
change\_GDPC\_pct 0.03697 og fortsatt signifikant, men kvadratet av
variabelen har ingen signifikant effekt. I den kubiske modellen er ingen
av de tre vekstrelaterte parameterne, change\_GDPC\_pct,
I(change\_GDPC\_pct\^{}2) eller I(change\_GDPC\_pct\^{}3), signifikante.
Dette viser at inkludering av tredjegradsleddet ikke gir ekstra
forklaringskraft, og modellen presterer svakere enn både log og
kvadratisk modell.

For det tredje er variabelen høy\_utdanning signifikant og positiv i
alle tre modeller, med koeffisienter rundt 0.00475 til 0.0045. Dette
viser at utdanningsnivået er den mest stabile forklaringsvariabelen i
modellene. Variabelen vei\_tetthet har negativ effekt i alle tre
modellene og er signifikant i både log og kvadratiske modeller, med
koeffisienter på henholdsvis -0.00087 og -0.00086. I den kubiske
modellen er effekten svakere og kun svakt signifikant, med koeffisient
-0.00078. Variabelen andelen\_over65 er ikke signifikant i noen av
modellspesifikasjonene.

Samlet sett viser tabell Tabell~\ref{tbl-funksjonsformer} at log lineær,
kvadratisk og kubisk modell alle fanger opp sammenhengen mellom
økonomisk utvikling og inntektsulikhet, men med ulike resultater. Log
lineær modell og kvadratisk modell viser tydeligere statistiske
sammenhenger for vekstvariabelen, mens den kubiske modellen, til tross
for høy R², ikke har signifikante vekstparametere og tilfører lite
informasjon. Basert på sammenligning av R² og AIC vurderes log lineær og
kvadratisk modell som de mest representative alternative funksjonelle
formene.

\subsubsection{Panelmodeller (Fire
FE-spesifikasjoner)}\label{panelmodeller-fire-fe-spesifikasjoner}

\begin{longtable}[]{@{}
  >{\raggedright\arraybackslash}p{(\linewidth - 8\tabcolsep) * \real{0.2361}}
  >{\raggedright\arraybackslash}p{(\linewidth - 8\tabcolsep) * \real{0.1667}}
  >{\raggedright\arraybackslash}p{(\linewidth - 8\tabcolsep) * \real{0.1667}}
  >{\raggedright\arraybackslash}p{(\linewidth - 8\tabcolsep) * \real{0.1806}}
  >{\raggedright\arraybackslash}p{(\linewidth - 8\tabcolsep) * \real{0.1806}}@{}}

\caption{\label{tbl-panel}Sammenligning av panelmodeller}

\tabularnewline

\toprule\noalign{}
\begin{minipage}[b]{\linewidth}\raggedright
\end{minipage} & \begin{minipage}[b]{\linewidth}\raggedright
Region FE
\end{minipage} & \begin{minipage}[b]{\linewidth}\raggedright
Year FE
\end{minipage} & \begin{minipage}[b]{\linewidth}\raggedright
Two-way FE
\end{minipage} & \begin{minipage}[b]{\linewidth}\raggedright
Country FE
\end{minipage} \\
\midrule\noalign{}
\endhead
\midrule\noalign{}
\multicolumn{5}{@{}>{\raggedright\arraybackslash}p{(\linewidth - 8\tabcolsep) * \real{0.9306} + 8\tabcolsep}@{}}{%
\begin{minipage}[t]{\linewidth}\raggedright
\begin{itemize}
\tightlist
\item
  p \textless{} 0.1, * p \textless{} 0.05, ** p \textless{} 0.01, *** p
  \textless{} 0.001
\end{itemize}
\end{minipage}} \\
\bottomrule\noalign{}
\endlastfoot
Høy\_utdanning & -0.00036 & 0.00011 & -0.00035 & 0.00041 \\
& (0.00044) & (0.00061) & (0.00053) & (0.00084) \\
andelen\_over65 & 0.00048 & 0.00315+ & 0.00096 & -0.00055 \\
& (0.00088) & (0.00185) & (0.00125) & (0.00155) \\
Num.Obs. & 204 & 204 & 204 & 35 \\
R2 & 0.004 & 0.022 & 0.006 & 0.009 \\
R2 Adj. & -0.130 & -0.028 & -0.180 & -0.162 \\
AIC & -1242.3 & -627.2 & -1259.9 & -247.4 \\
BIC & -1232.3 & -617.3 & -1249.9 & -242.8 \\
RMSE & 0.01 & 0.05 & 0.01 & 0.01 \\

\end{longtable}

For å analysere de dynamiske endringene innenfor regioner over tid
estimerer oppgave 3 fire typer faste effektsmodeller: region FE, year
FE, two way FE og country FE. Variabelen veitetthet er ikke inkludert i
panelmodellene, siden faste effekter kun kan estimere variabler som
varierer over tid innen samme region. Variabler som er tidsinvariante
eller har svært liten variasjon kan ikke få identifisert koeffisienter.
Derfor inneholder panelanalysen kun variablene høy\_utdanning og
andelen\_over65, som begge varierer over tid.

For det første viser tabell Tabell~\ref{tbl-panel} at alle
panelmodellene har svært lave R2 verdier. Region FE har R² på 0.004,
year FE 0.022, two way FE 0.006 og country FE 0.009. Den justerte R² er
negativ for alle modellene i området fra -0.130 til -0.028. Dette viser
at uansett hvilken type faste effekter som benyttes, har modellene svært
begrenset evne til å forklare variasjonen i Gini over tid.

For det andre viser koeffisientenes signifikans at nesten ingen
variabler er signifikante i noen av de fire modellene. Høy\_utdanning
ligger svært nær null i alle spesifikasjoner, med koeffisienter mellom
-0.00036 og 0.00041, og ingen av dem er signifikante. Andelen\_over65
viser svak signifikans i year FE modellen med koeffisient 0.00315, men
er ikke signifikant i de øvrige modellene. Dette betyr at både
utdanningsnivå og andelen eldre over 65 år har svært liten variasjon
over tid, og selv etter kontroll for faste effekter bidrar disse
variablene i liten grad til å forklare årlige endringer i Gini.

For det tredje viser modellsammenligningen at year FE modellen har den
høyeste R² med 0.022, men forklaringskraften er fortsatt svært lav. Two
way FE kontrollerer både region og år, men har likevel en R² på kun
0.006, som er lavere enn year FE. Country FE modellen bruker faste
effekter på landsnivå, men inkluderer bare 35 observasjoner, og ingen av
koeffisientene er signifikante.

Samlet sett viser tabell Tabell~\ref{tbl-panel} at alle fire
panelmodellene har svært svak forklaringskraft, og at nøkkelvariablene
ikke viser signifikante effekter i tidsdimensjonen. Koeffisientene
ligger nær null, og R² verdiene er gjennomgående lave. Dette viser at
faste effektsmodeller ikke klarer å identifisere tydelige dynamiske
endringer i regional inntektsulikhet over tid i dette datasettet.

\section{Diskusjon}\label{diskusjon}

\subsection{Viktige innsikter}\label{viktige-innsikter}

Gjennom en komplementær analyse av tverrsnittsmodeller, alternative
funksjonelle former og panelmodeller fremkommer det at regional
inntektsulikhet har en kompleks og flerlagd struktur, og at ulikhet ikke
bestemmes av en enkelt faktor, men av samspillet mellom flere regionale
kjennetegn.

For det første viser den enkle lineære regresjonsmodellen en positiv
sammenheng mellom økonomisk utvikling og regional ulikhet, men modellens
forklaringskraft er lav med en justert R² på om lag 25 prosent. Dette
viser at økonomisk vekst alene ikke er tilstrekkelig for å forklare
ulikhetsnivået mellom regioner.

For det andre viser den multiple lineære regresjonsmodellen en betydelig
økning i forklaringskraft, til rundt 49 prosent, når sentrale regionale
faktorer som utdanning (høy utdanning), transportinfrastruktur
(veitetthet) og demografisk struktur (andel av befolkningen som er 65 år
eller eldre) inkluderes. Høy utdanning har en signifikant positiv
effekt, noe som viser at utdanning er den viktigste
forklaringsvariabelen. Veitetthet har en signifikant negativ effekt, noe
som innebærer at regioner med bedre transportinfrastruktur har lavere
inntektsulikhet. Økonomisk utvikling beholder en positiv koeffisient,
men uten statistisk signifikans, noe som viser at dens forklaringsbidrag
er begrenset.

For det tredje viser de alternative funksjonelle formene at sammenhengen
mellom økonomisk utvikling og regional ulikhet varierer noe i
signifikans på tvers av modellspesifikasjoner. Selv om log modellen og
den kvadratiske modellen gir bedre modelltilpasning, er hovedresultatene
stabilt: høy utdanning og veitetthet er signifikante i alle modeller,
mens økonomisk utvikling viser svakere og mindre stabile effekter. Dette
viser at utdanning og infrastruktur gir mer robuste forklaringer på
ulikhet enn økonomisk vekst.

Til slutt viser panelmodellene at alle faste effektsmodeller har svært
lav forklaringskraft. Verken country FE, year FE, region FE eller two
way FE viser signifikante effekter for nøkkelvariablene over tid. Dette
gjenspeiler at regional ulikhet, utdanningsnivå og befolkningsstruktur i
perioden 2008 til 2022 endrer seg svært lite innen regioner, noe som
gjør det vanskelig for faste effekter å identifisere tydelige
tidsvariasjoner.

Samlet sett viser disse modellene at forskjeller i regional
inntektsulikhet hovedsakelig kommer fra langsiktige forskjeller mellom
regioner, særlig knyttet til utdanningsnivå og transportinfrastruktur,
og ikke fra endringer over tid. Høy utdanning og veitetthet er
signifikante i både tverrsnittsmodeller og de alternative funksjonelle
formene, mens variablene ikke viser signifikante effekter i
panelmodellene fordi de endrer seg svært lite over tid. Dette viser at
regional inntektsulikhet i stor grad gjenspeiler varige forskjeller
mellom regioner, og ikke variasjoner fra år til år.

\subsection{Politiske implikasjoner}\label{politiske-implikasjoner}

Resultatene fra oppgavene viser at utdanningsnivå og
transportinfrastruktur er de mest konsistente faktorene knyttet til
regional inntektsulikhet, mens økonomisk vekst viser svakere og mindre
stabile sammenhenger. Dette gir et grunnlag for å vurdere hvilke typer
tiltak som kan være mest målrettet når en ønsker å redusere regionale
forskjeller.

For det første er utdanningsnivået konsekvent signifikant i både
tverrsnittsmodellen og de alternative funksjonelle formene, og fremstår
som den viktigste forklaringsvariabelen. Dette innebærer at økt tilgang
til utdanning er ett av de mest effektive langsiktige virkemidlene for å
redusere regional ulikhet. I regioner med svake utdanningsressurser bør
politikken prioritere bedre tilgang til høyere utdanning og
kompetanseheving, for å styrke regionale utviklingsmuligheter.

For det andre viser transportinfrastruktur, målt ved veitetthet, en
tydelig negativ sammenheng med ulikhet på tvers av flere modeller. Dette
tyder på at bedre regional tilknytning kan fremme balansert utvikling.
For regioner med svak infrastruktur kan investeringer i
transportforbindelser være et viktig virkemiddel for å redusere
regionale forskjeller.

For det tredje er den økonomiske veksten ustabil i sin signifikans, og
forklaringskraften er betydelig svakere enn for utdanning og
infrastruktur. Dette betyr at økonomisk vekst alene ikke er et effektivt
virkemiddel for å redusere regionale forskjeller. Regionalpolitikken bør
derfor rette større oppmerksomhet mot strukturelle faktorer som
utdanning og infrastruktur, i stedet for kortsiktige vekststimulerende
tiltak.

Til slutt viser panelanalysen at de interne endringene innen regioner i
perioden 2008 til 2022 er svært begrensede, og at regionale forskjeller
har en tydelig langsiktig karakter. Dette innebærer at kortsiktige
tiltak vanskelig kan endre ulikhetsmønstre raskt, og at politikken
derfor må bygge på langsiktige og vedvarende strukturelle tiltak.

Med dette som bakgrunn kan politikk for å redusere regionale forskjeller
med fordel prioritere tiltak som styrker utdanningsmuligheter og
forbedrer transportinfrastruktur. Slike tiltak har større sannsynlighet
for å gi varige forbedringer enn kortsiktige økonomiske vekststimuli.

\section{Begrensninger og fremtidig
forskning}\label{begrensninger-og-fremtidig-forskning}

\subsection{Forskningsbegrensninger}\label{forskningsbegrensninger}

Oppgavene har flere begrensninger knyttet til data og metode. For det
første varierer Eurostats dekning mellom land, og sammenslåingen av
NUTS3 til NUTS2 medfører enkelte manglende verdier, noe som reduserer
antall observasjoner. Dette begrenser regresjonsmodellene og fører til
en ubalansert panelstruktur, som igjen svekker muligheten til å
identifisere tidsmessige endringer.

For det andre endrer de sentrale forklaringsvariablene seg svært lite
over tid innen regionene. Dette gjelder blant annet utdanningsnivå,
veitetthet og demografisk struktur. Dette gjør det vanskelig for faste
effektsmodeller å identifisere deres tidsdynamiske effekter, noe som
fører til generelt svak signifikans i panelanalysene.

For det tredje inkluderer oppgavene kun utdanningsnivå, veitetthet og
demografisk struktur, ikke andre faktorer som kan påvirke regional
ulikhet. På grunn av det begrensede antallet forklaringsvariabler kan
modellene bare delvis belyse de faktorene som driver regional ulikhet.

Til slutt bygger oppgavene sine konklusjoner på statistiske sammenhenger
observert i modellene, og forklaringskraften er begrenset av datadekning
og modellspesifikasjoner. Mer detaljerte mekanismer mellom variablene
krever videre forskning med bredere datagrunnlag og utvidede metodiske
tilnærminger.

\section{Konklusjon}\label{konklusjon}

\subsection{Sammendrag}\label{sammendrag}

Gjennom å kombinere den langsiktige deskriptive analysen i oppgave 1,
tverrsnittsregresjonsmodellen i oppgave 2 og de alternative funksjonelle
formene samt panelmodellene med faste effekter i oppgave 3, undersøker
de oppgavene systematisk regionale forskjeller i økonomisk utvikling og
sentrale faktorer som påvirker regional inntektsulikhet i Belgia,
Nederland, Bulgaria og Norge.

Tverrsnittsanalysen viser at høy utdanning og veitetthet har stabile og
signifikante effekter på tvers av ulike modellspecifikasjoner, og er de
viktigste forklaringsfaktorene for regional ulikhet. Høy utdanning har
en signifikant positiv sammenheng med inntektsulikhet, noe som innebærer
at regioner med en større andel høyt utdannede ofte har høyere ulikhet.
Veitetthet har derimot en negativ sammenheng med ulikhet, og regioner
med bedre transporttilgang har som regel lavere inntektsforskjeller.
Sammenlignet med dette viser endringen i BNP per innbygger ustabile
signifikansnivåer i flere modeller, noe som tyder på at økonomisk vekst
ikke er den viktigste forklaringsfaktoren bak regional ulikhet.

Resultatene fra de alternative funksjonelle formene bekrefter dette
bildet. Selv om log lineære og kvadratiske modeller gir en viss
forbedring i modelltilpasning, endres verken koeffisientenes retning
eller deres signifikans på en vesentlig måte. Effekten av høy utdanning
og veitetthet forblir robust, mens økonomisk vekst fortsatt har
begrenset forklaringskraft.

Panelmodellene viser at de sentrale variablene har svært begrenset
tidsvariasjon innen regionene i perioden 2008 til 2022. Høy utdanning og
den demografiske strukturen (andel 65 år eller eldre) fremstår uten
signifikante effekter i faste effektsmodeller. Dette indikerer at
regional inntektsulikhet i større grad gjenspeiler vedvarende
forskjeller mellom regioner enn kortsiktige årlige endringer.

Samlet sett viser modellresultatene at tiltak for å redusere regionale
inntektsforskjeller bør legge større vekt på langsiktige innsatsområder
som å øke utdanningsnivå og forbedre transportinfrastruktur, i stedet
for å støtte seg på kortsiktige tiltak for å stimulere økonomisk vekst.

Betydningen av disse funnene er at de bidrar til å identifisere hvilke
faktorer som har sterkest sammenheng med regional ulikhet, noe som kan
hjelpe beslutningstakere med å målrette innsatsen mer effektivt.

\subsection{Sluttrefleksjon}\label{sluttrefleksjon}

Basert på de ulike datastrukturene og metodiske tilnærmingene som
oppgave 1, 2 og 3 benytter, har forholdet mellom regional utvikling og
inntektsulikhet kunnet undersøkes fra flere perspektiver. Ved å
kombinere deskriptive analyser, tverrsnittsmodeller og panelestimeringer
har oppgavene vist hvordan valg av metode og datagrunnlag påvirker de
empiriske resultatene, og understreket betydningen av å bruke flere
metodiske tilnærminger i regional forskning.

Utfordringer knyttet til datadekning på tvers av land, manglende
observasjoner og begrenset tidsvariasjon i sentrale variabler
gjenspeiler også praktiske begrensninger som ofte oppstår i analyser av
regionale data. Til tross for disse utfordringene viser resultatene på
tvers av modellene en viss konsistens, noe som gjør det mulig å
identifisere utdanningsnivå og transportinfrastruktur som de mest
sentrale faktorene knyttet til regionale forskjeller.

Analysene gir et mer konkret bilde av hvordan et lite sett av variabler
kan være knyttet til regional inntektsulikhet. Selv om analysene ikke
kan si noe om årsakssammenhenger, gir arbeidet nyttige erfaringer med
bruk av ulike metoder, noe som kan være relevant for videre empiriske
studier på området.

\section{Vedlegg-Bruken av
AI-verktøy}\label{vedlegg-bruken-av-ai-verktuxf8y}

Jeg benyttet ChatGPT (modell GPT-5.1) i arbeidet med MSB104 oppgave 4,
hovedsakelig som støtte til oversettelse mellom kinesisk og norsk.
Basert på mine egne kinesiske utkast brukte jeg ChatGPT til å oversette
innholdet til norsk og til å få forslag til mer naturlige eller mer
akademiske formuleringer. Jeg gjennomgikk alle forslagene og teksten som
ble generert av ChatGPT, redigerte og omorganiserte dem for å sikre at
terminologien, strukturen og det analytiske rammeverket var i samsvar
med det opprinnelige utkastet. Gjennom hele prosessen bidro ChatGPT til
å forbedre sammenheng og språklig klarhet i tekstene. Alle formuleringer
ble kontrollert av meg setning for setning, og jeg sørget for at de var
i tråd med min egen analyse og forståelse.

\section*{Referanser}\label{referanser}
\addcontentsline{toc}{section}{Referanser}

\phantomsection\label{refs}
\begin{CSLReferences}{1}{0}
\bibitem[\citeproctext]{ref-su13021006}
Bandeira Morais, M., Swart, J., \& Jordaan, J. A. (2021). Economic
Complexity and Inequality: Does Regional Productive Structure Affect
Income Inequality in Brazilian States? \emph{Sustainability},
\emph{13}(2). \url{https://doi.org/10.3390/su13021006}

\bibitem[\citeproctext]{ref-Calderon2004Infrastructure}
Calderón, C., \& Servén, L. (2004). \emph{The Effects of Infrastructure
Development on Growth and Income Distribution} (Policy Research Working
Paper Nr. 3400). World Bank. \url{https://hdl.handle.net/10986/14136}

\bibitem[\citeproctext]{ref-Cappelen2009}
Cappelen, Ø., \& Mjøset, L. (2009). Can Norway Be a Role Model for
Natural Resource Abundant Countries? \emph{Discussion Papers in
Economics}, \emph{09}, 1--26.
\url{https://www.researchgate.net/publication/228983025_Can_Norway_Be_a_Role_Model_for_Natural_Resource_Abundant_Countries}

\bibitem[\citeproctext]{ref-Coady2017Income}
Coady, D., \& Dizioli, A. (2017). Income {Inequality} and {Education}
{Revisited}: Persistence, {Endogeneity}, and {Heterogeneity}. \emph{IMF
Working Papers}, \emph{17}(126), 1.
\url{https://doi.org/10.5089/9781475595741.001}

\bibitem[\citeproctext]{ref-dolls2019demographic}
Dolls, M., Doorley, K., Paulus, A., Schneider, H., \& Sommer, E. (2019).
Demographic change and the European income distribution. \emph{The
Journal of Economic Inequality}, \emph{17}(3), 337--357.

\bibitem[\citeproctext]{ref-kuznets2019economic}
Kuznets, S. (2019). Economic growth and income inequality. I \emph{The
gap between rich and poor} (s. 25--37). Routledge.

\bibitem[\citeproctext]{ref-LESSMANN2017110}
Lessmann, C., \& Seidel, A. (2017). Regional inequality, convergence,
and its determinants -- A view from outer space. \emph{European Economic
Review}, \emph{92}, 110--132.
https://doi.org/\url{https://doi.org/10.1016/j.euroecorev.2016.11.009}

\end{CSLReferences}




\end{document}
